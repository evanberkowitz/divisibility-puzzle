\section{Solution}

Because each digit is unique, we can easily display our current knowledge of constraints in a table which we hope to fill in using knowledge from the divisibility properties in the problems statement.
Let's fill in these constraints as best we can, and show every deduction.

Starting with a blank table, we use the easiest rule---to be divisible by 10 a number must end in a zero.
\begin{center}
\begin{tabular}{ccccccccccc}
			&	$a$	&	$b$	&	$c$	&	$d$	&	$e$	&	$f$	&	$g$	&	$h$	&	$i$	&	$j$	\\
	1		&		&		&		&		&		&		&		&		&		&		\\
	2		&		&		&		&		&		&		&		&		&		&		\\
	3		&		&		&		&		&		&		&		&		&		&		\\
	4		&		&		&		&		&		&		&		&		&		&		\\
	5		&		&		&		&		&		&		&		&		&		&		\\
	6		&		&		&		&		&		&		&		&		&		&		\\
	7		&		&		&		&		&		&		&		&		&		&		\\
	8		&		&		&		&		&		&		&		&		&		&		\\
	9		&		&		&		&		&		&		&		&		&		&		\\
	0		&		&		&		&		&		&		&		&		&		&		\\
\end{tabular}
\hspace{1cm}\vline\hspace{1cm}
\begin{tabular}{rl}
	& \\
	1	&	divides $a$,					\\
	2	&	divides $ab$,					\\
	3	&	divides $abc$,					\\
	4	&	divides $abcd$,					\\
	5	&	divides $abcde$,				\\
	6	&	divides $abcdef$,				\\
	7	&	divides $abcdefg$,				\\
	8	&	divides $abcdefgh$,				\\
	9	&	divides $abcdefghi$, and		\\
	10	&	divides $abcdefghij$ which implies $j=0$.			
\end{tabular}
\end{center}

Therefore we can update the table and use the divisibility-by-5 rule,
\begin{center}
\begin{tabular}{ccccccccccc}
			&	$a$	&	$b$	&	$c$	&	$d$	&	$e$	&	$f$	&	$g$	&	$h$	&	$i$	&	$j$	\\
	1		&		&		&		&		&		&		&		&		&		&	x	\\
	2		&		&		&		&		&		&		&		&		&		&	x	\\
	3		&		&		&		&		&		&		&		&		&		&	x	\\
	4		&		&		&		&		&		&		&		&		&		&	x	\\
	5		&		&		&		&		&		&		&		&		&		&	x	\\
	6		&		&		&		&		&		&		&		&		&		&	x	\\
	7		&		&		&		&		&		&		&		&		&		&	x	\\
	8		&		&		&		&		&		&		&		&		&		&	x	\\
	9		&		&		&		&		&		&		&		&		&		&	x	\\
	0		&	x	&	x	&	x	&	x	&	x	&	x	&	x	&	x	&	x	&	0	\\
\end{tabular}
\hspace{1cm}\vline\hspace{1cm}
\begin{tabular}{rll}
	& \\
	1	&	divides $a$						\\
	2	&	divides $ab$					\\
	3	&	divides $abc$					\\
	4	&	divides $abcd$					\\
	5	&	divides $abcde$				& implies $e=5$	\\
	6	&	divides $abcdef$				\\
	7	&	divides $abcdefg$				\\
	8	&	divides $abcdefgh$				\\
	9	&	divides $abcdefghi$				\\
		&	
\end{tabular}
\end{center}

Note that any integer is divisible by 1, so the first constraint doesn't help us at all.
\begin{center}
\begin{tabular}{ccccccccccc}
			&	$a$	&	$b$	&	$c$	&	$d$	&	$e$	&	$f$	&	$g$	&	$h$	&	$i$	&	$j$	\\
	1		&		&		&		&		&	x	&		&		&		&		&	x	\\
	2		&		&		&		&		&	x	&		&		&		&		&	x	\\
	3		&		&		&		&		&	x	&		&		&		&		&	x	\\
	4		&		&		&		&		&	x	&		&		&		&		&	x	\\
	5		&	x	&	x	&	x	&	x	&	5	&	x	&	x	&	x	&	x	&	x	\\
	6		&		&		&		&		&	x	&		&		&		&		&	x	\\
	7		&		&		&		&		&	x	&		&		&		&		&	x	\\
	8		&		&		&		&		&	x	&		&		&		&		&	x	\\
	9		&		&		&		&		&	x	&		&		&		&		&	x	\\
	0		&	x	&	x	&	x	&	x	&	x	&	x	&	x	&	x	&	x	&	0	\\
\end{tabular}
\hspace{1cm}\vline\hspace{1cm}
\begin{tabular}{rll}
	& 		\\
	1	&	divides $a$					& implies no constraint	\\
	2	&	divides $ab$					\\
	3	&	divides $abc$					\\
	4	&	divides $abcd$					\\
		&	\\
	6	&	divides $abcdef$				\\
	7	&	divides $abcdefg$				\\
	8	&	divides $abcdefgh$				\\
	9	&	divides $abcdefghi$				\\
		&	
\end{tabular}
\end{center}

The even rules mean that the corresponding final digits must be even.  This exhausts the even choices; the odd digits must be odd.
\begin{center}
\begin{tabular}{ccccccccccc}
			&	$a$	&	$b$	&	$c$	&	$d$	&	$e$	&	$f$	&	$g$	&	$h$	&	$i$	&	$j$	\\
	1		&		&	x	&		&	x	&	x	&	x	&		&	x	&		&	x	\\
	2		&	x	&		&	x	&		&	x	&		&	x	&		&	x	&	x	\\
	3		&		&	x	&		&	x	&	x	&	x	&		&	x	&		&	x	\\
	4		&	x	&		&	x	&		&	x	&		&	x	&		&	x	&	x	\\
	5		&	x	&	x	&	x	&	x	&	5	&	x	&	x	&	x	&	x	&	x	\\
	6		&	x	&		&	x	&		&	x	&		&	x	&		&	x	&	x	\\
	7		&		&	x	&		&	x	&	x	&	x	&		&	x	&		&	x	\\
	8		&	x	&		&	x	&		&	x	&		&	x	&		&	x	&	x	\\
	9		&		&	x	&		&	x	&	x	&	x	&		&	x	&		&	x	\\
	0		&	x	&	x	&	x	&	x	&	x	&	x	&	x	&	x	&	x	&	0	\\
\end{tabular}
\hspace{1cm}\vline\hspace{1cm}
\begin{tabular}{rll}
	& 		\\
	& 		\\
	2	&	divides $ab$					\\
	3	&	divides $abc$					\\
	4	&	divides $abcd$					\\
		&	\\
	6	&	divides $abcdef$				\\
	7	&	divides $abcdefg$				\\
	8	&	divides $abcdefgh$				\\
	9	&	divides $abcdefghi$				\\
		&	
\end{tabular}
\end{center}

Let's leverage our knowledge of divisibility rules to simplify,
\begin{center}
\begin{tabular}{ccccccccccc}
			&	$a$	&	$b$	&	$c$	&	$d$	&	$e$	&	$f$	&	$g$	&	$h$	&	$i$	&	$j$	\\
	1		&		&	x	&		&	x	&	x	&	x	&		&	x	&		&	x	\\
	2		&	x	&		&	x	&		&	x	&		&	x	&		&	x	&	x	\\
	3		&		&	x	&		&	x	&	x	&	x	&		&	x	&		&	x	\\
	4		&	x	&		&	x	&		&	x	&		&	x	&		&	x	&	x	\\
	5		&	x	&	x	&	x	&	x	&	5	&	x	&	x	&	x	&	x	&	x	\\
	6		&	x	&		&	x	&		&	x	&		&	x	&		&	x	&	x	\\
	7		&		&	x	&		&	x	&	x	&	x	&		&	x	&		&	x	\\
	8		&	x	&		&	x	&		&	x	&		&	x	&		&	x	&	x	\\
	9		&		&	x	&		&	x	&	x	&	x	&		&	x	&		&	x	\\
	0		&	x	&	x	&	x	&	x	&	x	&	x	&	x	&	x	&	x	&	0	\\
\end{tabular}
\hspace{1cm}\vline\hspace{1cm}
\begin{tabular}{rll}
	& 		\\
	& 		\\
	& 		\\
	3	&	divides $abc$			& implies 3 divides $a+b+c$					\\
	4	&	divides $abcd$			& implies 4 divides $cd$					\\
		&	\\
	6	&	divides $abcdef$		& implies 3 divides $a+b+c+d+e+f$			\\
	7	&	divides $abcdefg$				\\
	8	&	divides $abcdefgh$		& implies 8 divides $fgh$					\\
	9	&	divides $abcdefghi$		& implies 9 divides $a+b+c+d+e+f+g+h+i$		\\
		&	
\end{tabular}
\end{center}

So that our knowledge of the possibilities and constraints is
\begin{center}
\begin{tabular}{ccccccccccc}
			&	$a$	&	$b$	&	$c$	&	$d$	&	$e$	&	$f$	&	$g$	&	$h$	&	$i$	&	$j$	\\
	1		&		&	x	&		&	x	&	x	&	x	&		&	x	&		&	x	\\
	2		&	x	&		&	x	&		&	x	&		&	x	&		&	x	&	x	\\
	3		&		&	x	&		&	x	&	x	&	x	&		&	x	&		&	x	\\
	4		&	x	&		&	x	&		&	x	&		&	x	&		&	x	&	x	\\
	5		&	x	&	x	&	x	&	x	&	5	&	x	&	x	&	x	&	x	&	x	\\
	6		&	x	&		&	x	&		&	x	&		&	x	&		&	x	&	x	\\
	7		&		&	x	&		&	x	&	x	&	x	&		&	x	&		&	x	\\
	8		&	x	&		&	x	&		&	x	&		&	x	&		&	x	&	x	\\
	9		&		&	x	&		&	x	&	x	&	x	&		&	x	&		&	x	\\
	0		&	x	&	x	&	x	&	x	&	x	&	x	&	x	&	x	&	x	&	0	\\
\end{tabular}
\hspace{1cm}\vline\hspace{1cm}
\begin{tabular}{rll}
	& 		\\
	& 		\\
	& 		\\
	3	&	divides $a+b+c$					\\
	4	&	divides $cd$					\\
		&	\\
	3	&	divides $a+b+c+d+e+f$			\\
	7	&	divides $abcdefg$				\\
	8	&	divides $fgh$					\\
	9	&	divides $a+b+c+d+e+f+g+h+i$		\\
		&	
\end{tabular}
\end{center}

Since 3 divides $a+b+c$, if it is to also divide $a+b+c+d+e+f$, it must separately divide $d+e+f$.
\begin{center}
\begin{tabular}{ccccccccccc}
			&	$a$	&	$b$	&	$c$	&	$d$	&	$e$	&	$f$	&	$g$	&	$h$	&	$i$	&	$j$	\\
	1		&		&	x	&		&	x	&	x	&	x	&		&	x	&		&	x	\\
	2		&	x	&		&	x	&		&	x	&		&	x	&		&	x	&	x	\\
	3		&		&	x	&		&	x	&	x	&	x	&		&	x	&		&	x	\\
	4		&	x	&		&	x	&		&	x	&		&	x	&		&	x	&	x	\\
	5		&	x	&	x	&	x	&	x	&	5	&	x	&	x	&	x	&	x	&	x	\\
	6		&	x	&		&	x	&		&	x	&		&	x	&		&	x	&	x	\\
	7		&		&	x	&		&	x	&	x	&	x	&		&	x	&		&	x	\\
	8		&	x	&		&	x	&		&	x	&		&	x	&		&	x	&	x	\\
	9		&		&	x	&		&	x	&	x	&	x	&		&	x	&		&	x	\\
	0		&	x	&	x	&	x	&	x	&	x	&	x	&	x	&	x	&	x	&	0	\\
\end{tabular}
\hspace{1cm}\vline\hspace{1cm}
\begin{tabular}{rll}
	& 		\\
	& 		\\
	& 		\\
	3	&	divides $a+b+c$					\\
	4	&	divides $cd$					\\
		&	\\
	3	&	divides $d+e+f$					\\
	7	&	divides $abcdefg$				\\
	8	&	divides $fgh$					\\
	9	&	divides $a+b+c+d+e+f+g+h+i$		\\
		&	
\end{tabular}
\end{center}

The sum of all the digits 0-9 is 45.  That makes the 9th rule look empty.  But knowing that 3 divides $a+b+c$ and $d+e+f$ we can conclude that 3 also divides $g+h+i$.
\begin{center}
\begin{tabular}{ccccccccccc}
			&	$a$	&	$b$	&	$c$	&	$d$	&	$e$	&	$f$	&	$g$	&	$h$	&	$i$	&	$j$	\\
	1		&		&	x	&		&	x	&	x	&	x	&		&	x	&		&	x	\\
	2		&	x	&		&	x	&		&	x	&		&	x	&		&	x	&	x	\\
	3		&		&	x	&		&	x	&	x	&	x	&		&	x	&		&	x	\\
	4		&	x	&		&	x	&		&	x	&		&	x	&		&	x	&	x	\\
	5		&	x	&	x	&	x	&	x	&	5	&	x	&	x	&	x	&	x	&	x	\\
	6		&	x	&		&	x	&		&	x	&		&	x	&		&	x	&	x	\\
	7		&		&	x	&		&	x	&	x	&	x	&		&	x	&		&	x	\\
	8		&	x	&		&	x	&		&	x	&		&	x	&		&	x	&	x	\\
	9		&		&	x	&		&	x	&	x	&	x	&		&	x	&		&	x	\\
	0		&	x	&	x	&	x	&	x	&	x	&	x	&	x	&	x	&	x	&	0	\\
\end{tabular}
\hspace{1cm}\vline\hspace{1cm}
\begin{tabular}{rll}
	& 		\\
	& 		\\
	& 		\\
	3	&	divides $a+b+c$					\\
	4	&	divides $cd$					\\
		&	\\
	3	&	divides $d+e+f$					\\
	7	&	divides $abcdefg$				\\
	8	&	divides $fgh$					\\
	3	&	divides $g+h+i$					\\
		&	
\end{tabular}
\end{center}

Since $cd$ is divisible by 4 and starts with an odd digit, it must be one of 12, 16, 32, 36, 72, 76, 92, or 96.
\begin{center}
\begin{tabular}{ccccccccccc}
			&	$a$	&	$b$	&	$c$	&	$d$	&	$e$	&	$f$	&	$g$	&	$h$	&	$i$	&	$j$	\\
	1		&		&	x	&		&	x	&	x	&	x	&		&	x	&		&	x	\\
	2		&	x	&		&	x	&		&	x	&		&	x	&		&	x	&	x	\\
	3		&		&	x	&		&	x	&	x	&	x	&		&	x	&		&	x	\\
	4		&	x	&		&	x	&		&	x	&		&	x	&		&	x	&	x	\\
	5		&	x	&	x	&	x	&	x	&	5	&	x	&	x	&	x	&	x	&	x	\\
	6		&	x	&		&	x	&		&	x	&		&	x	&		&	x	&	x	\\
	7		&		&	x	&		&	x	&	x	&	x	&		&	x	&		&	x	\\
	8		&	x	&		&	x	&		&	x	&		&	x	&		&	x	&	x	\\
	9		&		&	x	&		&	x	&	x	&	x	&		&	x	&		&	x	\\
	0		&	x	&	x	&	x	&	x	&	x	&	x	&	x	&	x	&	x	&	0	\\
\end{tabular}
\hspace{1cm}\vline\hspace{1cm}
\begin{tabular}{rll}
	& 		\\
	& 		\\
	& 		\\
	3	&	divides $a+b+c$					\\
	4	&	divides $cd$				& implies $d$ is 2 or 6	\\
		&	\\
	3	&	divides $d+e+f$					\\
	7	&	divides $abcdefg$				\\
	8	&	divides $fgh$					\\
	3	&	divides $g+h+i$					\\
		&	
\end{tabular}
\end{center}


