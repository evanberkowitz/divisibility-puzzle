\section{Solution}

Because each digit is unique, we can easily display our current knowledge of constraints in a table which we hope to fill in using knowledge from the divisibility properties in the problems statement.
Let's fill in these constraints as best we can, and show every deduction.

Starting with a blank table, we use the easiest rule---to be divisible by 10 a number must end in a zero.
\begin{center}
\begin{tabular}{ccccccccccc}
			&	$a$	&	$b$	&	$c$	&	$d$	&	$e$	&	$f$	&	$g$	&	$h$	&	$i$	&	$j$	\\
	1		&		&		&		&		&		&		&		&		&		&		\\
	2		&		&		&		&		&		&		&		&		&		&		\\
	3		&		&		&		&		&		&		&		&		&		&		\\
	4		&		&		&		&		&		&		&		&		&		&		\\
	5		&		&		&		&		&		&		&		&		&		&		\\
	6		&		&		&		&		&		&		&		&		&		&		\\
	7		&		&		&		&		&		&		&		&		&		&		\\
	8		&		&		&		&		&		&		&		&		&		&		\\
	9		&		&		&		&		&		&		&		&		&		&		\\
	0		&		&		&		&		&		&		&		&		&		&		\\
\end{tabular}
\hspace{1cm}\vline\hspace{1cm}
\begin{tabular}{rl}
	& \\
	1	&	divides $a$,					\\
	2	&	divides $ab$,					\\
	3	&	divides $abc$,					\\
	4	&	divides $abcd$,					\\
	5	&	divides $abcde$,				\\
	6	&	divides $abcdef$,				\\
	7	&	divides $abcdefg$,				\\
	8	&	divides $abcdefgh$,				\\
	9	&	divides $abcdefghi$, and		\\
	10	&	divides $abcdefghij$ which implies $j=0$.			
\end{tabular}
\end{center}

Therefore we can update the table and use the divisibility-by-5 rule,
\begin{center}
\begin{tabular}{ccccccccccc}
			&	$a$	&	$b$	&	$c$	&	$d$	&	$e$	&	$f$	&	$g$	&	$h$	&	$i$	&	$j$	\\
	1		&		&		&		&		&		&		&		&		&		&	x	\\
	2		&		&		&		&		&		&		&		&		&		&	x	\\
	3		&		&		&		&		&		&		&		&		&		&	x	\\
	4		&		&		&		&		&		&		&		&		&		&	x	\\
	5		&		&		&		&		&		&		&		&		&		&	x	\\
	6		&		&		&		&		&		&		&		&		&		&	x	\\
	7		&		&		&		&		&		&		&		&		&		&	x	\\
	8		&		&		&		&		&		&		&		&		&		&	x	\\
	9		&		&		&		&		&		&		&		&		&		&	x	\\
	0		&	x	&	x	&	x	&	x	&	x	&	x	&	x	&	x	&	x	&	0	\\
\end{tabular}
\hspace{1cm}\vline\hspace{1cm}
\begin{tabular}{rll}
	& \\
	1	&	divides $a$						\\
	2	&	divides $ab$					\\
	3	&	divides $abc$					\\
	4	&	divides $abcd$					\\
	5	&	divides $abcde$				& implies $e=5$	\\
	6	&	divides $abcdef$				\\
	7	&	divides $abcdefg$				\\
	8	&	divides $abcdefgh$				\\
	9	&	divides $abcdefghi$				\\
		&	
\end{tabular}
\end{center}

Note that any integer is divisible by 1, so the first constraint doesn't help us at all.
\begin{center}
\begin{tabular}{ccccccccccc}
			&	$a$	&	$b$	&	$c$	&	$d$	&	$e$	&	$f$	&	$g$	&	$h$	&	$i$	&	$j$	\\
	1		&		&		&		&		&	x	&		&		&		&		&	x	\\
	2		&		&		&		&		&	x	&		&		&		&		&	x	\\
	3		&		&		&		&		&	x	&		&		&		&		&	x	\\
	4		&		&		&		&		&	x	&		&		&		&		&	x	\\
	5		&	x	&	x	&	x	&	x	&	5	&	x	&	x	&	x	&	x	&	x	\\
	6		&		&		&		&		&	x	&		&		&		&		&	x	\\
	7		&		&		&		&		&	x	&		&		&		&		&	x	\\
	8		&		&		&		&		&	x	&		&		&		&		&	x	\\
	9		&		&		&		&		&	x	&		&		&		&		&	x	\\
	0		&	x	&	x	&	x	&	x	&	x	&	x	&	x	&	x	&	x	&	0	\\
\end{tabular}
\hspace{1cm}\vline\hspace{1cm}
\begin{tabular}{rll}
	& 		\\
	1	&	divides $a$					& implies no constraint	\\
	2	&	divides $ab$					\\
	3	&	divides $abc$					\\
	4	&	divides $abcd$					\\
		&	\\
	6	&	divides $abcdef$				\\
	7	&	divides $abcdefg$				\\
	8	&	divides $abcdefgh$				\\
	9	&	divides $abcdefghi$				\\
		&	
\end{tabular}
\end{center}

The even rules mean that the corresponding final digits must be even.  This exhausts the even choices; the odd digits must be odd.
\begin{center}
\begin{tabular}{ccccccccccc}
			&	$a$	&	$b$	&	$c$	&	$d$	&	$e$	&	$f$	&	$g$	&	$h$	&	$i$	&	$j$	\\
	1		&		&	x	&		&	x	&	x	&	x	&		&	x	&		&	x	\\
	2		&	x	&		&	x	&		&	x	&		&	x	&		&	x	&	x	\\
	3		&		&	x	&		&	x	&	x	&	x	&		&	x	&		&	x	\\
	4		&	x	&		&	x	&		&	x	&		&	x	&		&	x	&	x	\\
	5		&	x	&	x	&	x	&	x	&	5	&	x	&	x	&	x	&	x	&	x	\\
	6		&	x	&		&	x	&		&	x	&		&	x	&		&	x	&	x	\\
	7		&		&	x	&		&	x	&	x	&	x	&		&	x	&		&	x	\\
	8		&	x	&		&	x	&		&	x	&		&	x	&		&	x	&	x	\\
	9		&		&	x	&		&	x	&	x	&	x	&		&	x	&		&	x	\\
	0		&	x	&	x	&	x	&	x	&	x	&	x	&	x	&	x	&	x	&	0	\\
\end{tabular}
\hspace{1cm}\vline\hspace{1cm}
\begin{tabular}{rll}
	& 		\\
	& 		\\
	2	&	divides $ab$					\\
	3	&	divides $abc$					\\
	4	&	divides $abcd$					\\
		&	\\
	6	&	divides $abcdef$				\\
	7	&	divides $abcdefg$				\\
	8	&	divides $abcdefgh$				\\
	9	&	divides $abcdefghi$				\\
		&	
\end{tabular}
\end{center}

Let's leverage our knowledge of divisibility rules to simplify,
\begin{center}
\begin{tabular}{ccccccccccc}
			&	$a$	&	$b$	&	$c$	&	$d$	&	$e$	&	$f$	&	$g$	&	$h$	&	$i$	&	$j$	\\
	1		&		&	x	&		&	x	&	x	&	x	&		&	x	&		&	x	\\
	2		&	x	&		&	x	&		&	x	&		&	x	&		&	x	&	x	\\
	3		&		&	x	&		&	x	&	x	&	x	&		&	x	&		&	x	\\
	4		&	x	&		&	x	&		&	x	&		&	x	&		&	x	&	x	\\
	5		&	x	&	x	&	x	&	x	&	5	&	x	&	x	&	x	&	x	&	x	\\
	6		&	x	&		&	x	&		&	x	&		&	x	&		&	x	&	x	\\
	7		&		&	x	&		&	x	&	x	&	x	&		&	x	&		&	x	\\
	8		&	x	&		&	x	&		&	x	&		&	x	&		&	x	&	x	\\
	9		&		&	x	&		&	x	&	x	&	x	&		&	x	&		&	x	\\
	0		&	x	&	x	&	x	&	x	&	x	&	x	&	x	&	x	&	x	&	0	\\
\end{tabular}
\hspace{1cm}\vline\hspace{1cm}
\begin{tabular}{rll}
	& 		\\
	& 		\\
	& 		\\
	3	&	divides $abc$			& implies 3 divides $a+b+c$					\\
	4	&	divides $abcd$			& implies 4 divides $cd$					\\
		&	\\
	6	&	divides $abcdef$		& implies 3 divides $a+b+c+d+e+f$			\\
	7	&	divides $abcdefg$				\\
	8	&	divides $abcdefgh$		& implies 8 divides $fgh$					\\
	9	&	divides $abcdefghi$		& implies 9 divides $a+b+c+d+e+f+g+h+i$		\\
		&	
\end{tabular}
\end{center}

So that our knowledge of the possibilities and constraints is
\begin{center}
\begin{tabular}{ccccccccccc}
			&	$a$	&	$b$	&	$c$	&	$d$	&	$e$	&	$f$	&	$g$	&	$h$	&	$i$	&	$j$	\\
	1		&		&	x	&		&	x	&	x	&	x	&		&	x	&		&	x	\\
	2		&	x	&		&	x	&		&	x	&		&	x	&		&	x	&	x	\\
	3		&		&	x	&		&	x	&	x	&	x	&		&	x	&		&	x	\\
	4		&	x	&		&	x	&		&	x	&		&	x	&		&	x	&	x	\\
	5		&	x	&	x	&	x	&	x	&	5	&	x	&	x	&	x	&	x	&	x	\\
	6		&	x	&		&	x	&		&	x	&		&	x	&		&	x	&	x	\\
	7		&		&	x	&		&	x	&	x	&	x	&		&	x	&		&	x	\\
	8		&	x	&		&	x	&		&	x	&		&	x	&		&	x	&	x	\\
	9		&		&	x	&		&	x	&	x	&	x	&		&	x	&		&	x	\\
	0		&	x	&	x	&	x	&	x	&	x	&	x	&	x	&	x	&	x	&	0	\\
\end{tabular}
\hspace{1cm}\vline\hspace{1cm}
\begin{tabular}{rll}
	& 		\\
	& 		\\
	& 		\\
	3	&	divides $a+b+c$					\\
	4	&	divides $cd$					\\
		&	\\
	3	&	divides $a+b+c+d+e+f$			\\
	7	&	divides $abcdefg$				\\
	8	&	divides $fgh$					\\
	9	&	divides $a+b+c+d+e+f+g+h+i$		\\
		&	
\end{tabular}
\end{center}

Since 3 divides $a+b+c$, if it is to also divide $a+b+c+d+e+f$, it must separately divide $d+e+f$.
\begin{center}
\begin{tabular}{ccccccccccc}
			&	$a$	&	$b$	&	$c$	&	$d$	&	$e$	&	$f$	&	$g$	&	$h$	&	$i$	&	$j$	\\
	1		&		&	x	&		&	x	&	x	&	x	&		&	x	&		&	x	\\
	2		&	x	&		&	x	&		&	x	&		&	x	&		&	x	&	x	\\
	3		&		&	x	&		&	x	&	x	&	x	&		&	x	&		&	x	\\
	4		&	x	&		&	x	&		&	x	&		&	x	&		&	x	&	x	\\
	5		&	x	&	x	&	x	&	x	&	5	&	x	&	x	&	x	&	x	&	x	\\
	6		&	x	&		&	x	&		&	x	&		&	x	&		&	x	&	x	\\
	7		&		&	x	&		&	x	&	x	&	x	&		&	x	&		&	x	\\
	8		&	x	&		&	x	&		&	x	&		&	x	&		&	x	&	x	\\
	9		&		&	x	&		&	x	&	x	&	x	&		&	x	&		&	x	\\
	0		&	x	&	x	&	x	&	x	&	x	&	x	&	x	&	x	&	x	&	0	\\
\end{tabular}
\hspace{1cm}\vline\hspace{1cm}
\begin{tabular}{rll}
	& 		\\
	& 		\\
	& 		\\
	3	&	divides $a+b+c$					\\
	4	&	divides $cd$					\\
		&	\\
	3	&	divides $d+e+f$					\\
	7	&	divides $abcdefg$				\\
	8	&	divides $fgh$					\\
	9	&	divides $a+b+c+d+e+f+g+h+i$		\\
		&	
\end{tabular}
\end{center}

The sum of all the digits 0-9 is 45.  That makes the 9th rule look empty.  But knowing that 3 divides $a+b+c$ and $d+e+f$ we can conclude that 3 also divides $g+h+i$.
\begin{center}
\begin{tabular}{ccccccccccc}
			&	$a$	&	$b$	&	$c$	&	$d$	&	$e$	&	$f$	&	$g$	&	$h$	&	$i$	&	$j$	\\
	1		&		&	x	&		&	x	&	x	&	x	&		&	x	&		&	x	\\
	2		&	x	&		&	x	&		&	x	&		&	x	&		&	x	&	x	\\
	3		&		&	x	&		&	x	&	x	&	x	&		&	x	&		&	x	\\
	4		&	x	&		&	x	&		&	x	&		&	x	&		&	x	&	x	\\
	5		&	x	&	x	&	x	&	x	&	5	&	x	&	x	&	x	&	x	&	x	\\
	6		&	x	&		&	x	&		&	x	&		&	x	&		&	x	&	x	\\
	7		&		&	x	&		&	x	&	x	&	x	&		&	x	&		&	x	\\
	8		&	x	&		&	x	&		&	x	&		&	x	&		&	x	&	x	\\
	9		&		&	x	&		&	x	&	x	&	x	&		&	x	&		&	x	\\
	0		&	x	&	x	&	x	&	x	&	x	&	x	&	x	&	x	&	x	&	0	\\
\end{tabular}
\hspace{1cm}\vline\hspace{1cm}
\begin{tabular}{rll}
	& 		\\
	& 		\\
	& 		\\
	3	&	divides $a+b+c$					\\
	4	&	divides $cd$					\\
		&	\\
	3	&	divides $d+e+f$					\\
	7	&	divides $abcdefg$				\\
	8	&	divides $fgh$					\\
	3	&	divides $g+h+i$					\\
		&	
\end{tabular}
\end{center}

Since $cd$ is divisible by 4 and starts with an odd digit, it must be one of 12, 16, 32, 36, 72, 76, 92, or 96.
\begin{center}
\begin{tabular}{ccccccccccc}
			&	$a$	&	$b$	&	$c$	&	$d$	&	$e$	&	$f$	&	$g$	&	$h$	&	$i$	&	$j$	\\
	1		&		&	x	&		&	x	&	x	&	x	&		&	x	&		&	x	\\
	2		&	x	&		&	x	&		&	x	&		&	x	&		&	x	&	x	\\
	3		&		&	x	&		&	x	&	x	&	x	&		&	x	&		&	x	\\
	4		&	x	&		&	x	&		&	x	&		&	x	&		&	x	&	x	\\
	5		&	x	&	x	&	x	&	x	&	5	&	x	&	x	&	x	&	x	&	x	\\
	6		&	x	&		&	x	&		&	x	&		&	x	&		&	x	&	x	\\
	7		&		&	x	&		&	x	&	x	&	x	&		&	x	&		&	x	\\
	8		&	x	&		&	x	&		&	x	&		&	x	&		&	x	&	x	\\
	9		&		&	x	&		&	x	&	x	&	x	&		&	x	&		&	x	\\
	0		&	x	&	x	&	x	&	x	&	x	&	x	&	x	&	x	&	x	&	0	\\
\end{tabular}
\hspace{1cm}\vline\hspace{1cm}
\begin{tabular}{rll}
	& 		\\
	& 		\\
	& 		\\
	3	&	divides $a+b+c$					\\
	4	&	divides $cd$				& implies $d$ is 2 or 6	\\
		&	\\
	3	&	divides $d+e+f$					\\
	7	&	divides $abcdefg$				\\
	8	&	divides $fgh$					\\
	3	&	divides $g+h+i$					\\
		&	
\end{tabular}
\end{center}
Since all $c$ are allowed, we have now entirely spent our divisible-by-4 rule.

With that knowledge of $d$, we see that
\begin{center}
\begin{tabular}{ccccccccccc}
			&	$a$	&	$b$	&	$c$	&	$d$	&	$e$	&	$f$	&	$g$	&	$h$	&	$i$	&	$j$	\\
	1		&		&	x	&		&	x	&	x	&	x	&		&	x	&		&	x	\\
	2		&	x	&		&	x	&		&	x	&		&	x	&		&	x	&	x	\\
	3		&		&	x	&		&	x	&	x	&	x	&		&	x	&		&	x	\\
	4		&	x	&		&	x	&	x	&	x	&		&	x	&		&	x	&	x	\\
	5		&	x	&	x	&	x	&	x	&	5	&	x	&	x	&	x	&	x	&	x	\\
	6		&	x	&		&	x	&		&	x	&		&	x	&		&	x	&	x	\\
	7		&		&	x	&		&	x	&	x	&	x	&		&	x	&		&	x	\\
	8		&	x	&		&	x	&	x	&	x	&		&	x	&		&	x	&	x	\\
	9		&		&	x	&		&	x	&	x	&	x	&		&	x	&		&	x	\\
	0		&	x	&	x	&	x	&	x	&	x	&	x	&	x	&	x	&	x	&	0	\\
\end{tabular}
\hspace{1cm}\vline\hspace{1cm}
\begin{tabular}{rll}
	& 		\\
	& 		\\
	& 		\\
	3	&	divides $a+b+c$					\\
	& 		\\
	&		\\
	3	&	divides $d+e+f$			& implies $def$ is 258 or 654		\\
	7	&	divides $abcdefg$				\\
	8	&	divides $fgh$					\\
	3	&	divides $g+h+i$					\\
		&	
\end{tabular}
\end{center}

That entails $f=4$ or $f=8$, both of which imply that the hundreds in $fgh$ is a multiple of 400, which is divisible by 8.  Therefore,
\begin{center}
\begin{tabular}{ccccccccccc}
			&	$a$	&	$b$	&	$c$	&	$d$	&	$e$	&	$f$	&	$g$	&	$h$	&	$i$	&	$j$	\\
	1		&		&	x	&		&	x	&	x	&	x	&		&	x	&		&	x	\\
	2		&	x	&		&	x	&		&	x	&	x	&	x	&		&	x	&	x	\\
	3		&		&	x	&		&	x	&	x	&	x	&		&	x	&		&	x	\\
	4		&	x	&		&	x	&	x	&	x	&		&	x	&		&	x	&	x	\\
	5		&	x	&	x	&	x	&	x	&	5	&	x	&	x	&	x	&	x	&	x	\\
	6		&	x	&		&	x	&		&	x	&	x	&	x	&		&	x	&	x	\\
	7		&		&	x	&		&	x	&	x	&	x	&		&	x	&		&	x	\\
	8		&	x	&		&	x	&	x	&	x	&		&	x	&		&	x	&	x	\\
	9		&		&	x	&		&	x	&	x	&	x	&		&	x	&		&	x	\\
	0		&	x	&	x	&	x	&	x	&	x	&	x	&	x	&	x	&	x	&	0	\\
\end{tabular}
\hspace{1cm}\vline\hspace{1cm}
\begin{tabular}{rll}
	& 		\\
	& 		\\
	& 		\\
	3	&	divides $a+b+c$		\\
	& 		\\
	&		\\
	&	$def$ is 258 or 654		\\
	7	&	divides $abcdefg$	\\
	8	&	divides $fgh$		& implies 8 divides $gh$	\\
	3	&	divides $g+h+i$		\\
		&	
\end{tabular}
\end{center}

Now we can reuse a trick from before.  If $gh$ is divisible by 8, then given the allowed values of $g$, it must be one of 16, 32, 72, or 96.
\begin{center}
\begin{tabular}{ccccccccccc}
			&	$a$	&	$b$	&	$c$	&	$d$	&	$e$	&	$f$	&	$g$	&	$h$	&	$i$	&	$j$	\\
	1		&		&	x	&		&	x	&	x	&	x	&		&	x	&		&	x	\\
	2		&	x	&		&	x	&		&	x	&	x	&	x	&		&	x	&	x	\\
	3		&		&	x	&		&	x	&	x	&	x	&		&	x	&		&	x	\\
	4		&	x	&		&	x	&	x	&	x	&		&	x	&		&	x	&	x	\\
	5		&	x	&	x	&	x	&	x	&	5	&	x	&	x	&	x	&	x	&	x	\\
	6		&	x	&		&	x	&		&	x	&	x	&	x	&		&	x	&	x	\\
	7		&		&	x	&		&	x	&	x	&	x	&		&	x	&		&	x	\\
	8		&	x	&		&	x	&	x	&	x	&		&	x	&		&	x	&	x	\\
	9		&		&	x	&		&	x	&	x	&	x	&		&	x	&		&	x	\\
	0		&	x	&	x	&	x	&	x	&	x	&	x	&	x	&	x	&	x	&	0	\\
\end{tabular}
\hspace{1cm}\vline\hspace{1cm}
\begin{tabular}{rll}
	& 		\\
	& 		\\
	& 		\\
	3	&	divides $a+b+c$		\\
	& 		\\
	&		\\
	&	$def$ is 258 or 654		\\
	7	&	divides $abcdefg$	\\
	8	&	divides $gh$		& implies $h$ is either 2 or 6 \\
	3	&	divides $g+h+i$		\\
		&	
\end{tabular}
\end{center}

Additionally, that means that $d$ and $h$ exhaust 2 and 6; so $b$ and $f$ must exhaust 4 and 8
\begin{center}
\begin{tabular}{ccccccccccc}
			&	$a$	&	$b$	&	$c$	&	$d$	&	$e$	&	$f$	&	$g$	&	$h$	&	$i$	&	$j$	\\
	1		&		&	x	&		&	x	&	x	&	x	&		&	x	&		&	x	\\
	2		&	x	&	x	&	x	&		&	x	&	x	&	x	&		&	x	&	x	\\
	3		&		&	x	&		&	x	&	x	&	x	&		&	x	&		&	x	\\
	4		&	x	&		&	x	&	x	&	x	&		&	x	&	x	&	x	&	x	\\
	5		&	x	&	x	&	x	&	x	&	5	&	x	&	x	&	x	&	x	&	x	\\
	6		&	x	&	x	&	x	&		&	x	&	x	&	x	&		&	x	&	x	\\
	7		&		&	x	&		&	x	&	x	&	x	&		&	x	&		&	x	\\
	8		&	x	&		&	x	&	x	&	x	&		&	x	&	x	&	x	&	x	\\
	9		&		&	x	&		&	x	&	x	&	x	&		&	x	&		&	x	\\
	0		&	x	&	x	&	x	&	x	&	x	&	x	&	x	&	x	&	x	&	0	\\
\end{tabular}
\hspace{1cm}\vline\hspace{1cm}
\begin{tabular}{rll}
	& 		\\
	& 		\\
	& 		\\
	3	&	divides $a+b+c$		\\
	& 		\\
	&		\\
	&	$def$ is 258 or 654		\\
	7	&	divides $abcdefg$	\\
		&	$gh$ is one of 16, 32, 72, or 96 \\
	3	&	divides $g+h+i$		\\
		&	
\end{tabular}
\end{center}

We can consider what the restrictions on $gh$ and the remaining possibilities of $i$ are divisible by 3,
\begin{center}
\begin{tabular}{ccccccccccc}
			&	$a$	&	$b$	&	$c$	&	$d$	&	$e$	&	$f$	&	$g$	&	$h$	&	$i$	&	$j$	\\
	1		&		&	x	&		&	x	&	x	&	x	&		&	x	&		&	x	\\
	2		&	x	&	x	&	x	&		&	x	&	x	&	x	&		&	x	&	x	\\
	3		&		&	x	&		&	x	&	x	&	x	&		&	x	&		&	x	\\
	4		&	x	&		&	x	&	x	&	x	&		&	x	&	x	&	x	&	x	\\
	5		&	x	&	x	&	x	&	x	&	5	&	x	&	x	&	x	&	x	&	x	\\
	6		&	x	&	x	&	x	&		&	x	&	x	&	x	&		&	x	&	x	\\
	7		&		&	x	&		&	x	&	x	&	x	&		&	x	&		&	x	\\
	8		&	x	&		&	x	&	x	&	x	&		&	x	&	x	&	x	&	x	\\
	9		&		&	x	&		&	x	&	x	&	x	&		&	x	&		&	x	\\
	0		&	x	&	x	&	x	&	x	&	x	&	x	&	x	&	x	&	x	&	0	\\
\end{tabular}
\hspace{1cm}\vline\hspace{1cm}
\begin{tabular}{rll}
	& 		\\
	& 		\\
	& 		\\
	3	&	divides $a+b+c$		\\
	& 		\\
	&		\\
	&	$def$ is 258 or 654		\\
	7	&	divides $abcdefg$	\\
		&	$gh$ is one of 16, 32, 72, or 96 & implies $ghi$ is one of 321, 327, 723, 729, and 963 \\
	3	&	divides $g+h+i$		\\
		&	
\end{tabular}
\end{center}

\begin{center}
\begin{tabular}{ccccccccccc}
			&	$a$	&	$b$	&	$c$	&	$d$	&	$e$	&	$f$	&	$g$	&	$h$	&	$i$	&	$j$	\\
	1		&		&	x	&		&	x	&	x	&	x	&		&	x	&		&	x	\\
	2		&	x	&	x	&	x	&		&	x	&	x	&	x	&		&	x	&	x	\\
	3		&		&	x	&		&	x	&	x	&	x	&		&	x	&		&	x	\\
	4		&	x	&		&	x	&	x	&	x	&		&	x	&	x	&	x	&	x	\\
	5		&	x	&	x	&	x	&	x	&	5	&	x	&	x	&	x	&	x	&	x	\\
	6		&	x	&	x	&	x	&		&	x	&	x	&	x	&		&	x	&	x	\\
	7		&		&	x	&		&	x	&	x	&	x	&		&	x	&		&	x	\\
	8		&	x	&		&	x	&	x	&	x	&		&	x	&	x	&	x	&	x	\\
	9		&		&	x	&		&	x	&	x	&	x	&		&	x	&		&	x	\\
	0		&	x	&	x	&	x	&	x	&	x	&	x	&	x	&	x	&	x	&	0	\\
\end{tabular}
\hspace{1cm}\vline\hspace{1cm}
\begin{tabular}{rll}
	& 		\\
	& 		\\
	& 		\\
	3	&	divides $a+b+c$		\\
	& 		\\
	&		\\
	&	$def$ is 258 or 654		\\
	7	&	divides $abcdefg$	\\
		&	$ghi$ is one of 321, 327, 723, 729, and 963 \\
		&	
		&	
\end{tabular}
\end{center}

Now we have boiled things down as far as possible.  We proceed by guess-and-check (or, if you prefer, proof-by-contradiction)

Suppose $def=258$.  Then $ghi=963$, the only choice without a 2, and $b=4$.  The divisibility-by-7 rule is for a number $a4c2589$.  Since $i=3$ the only choices for $a$ and $c$ are 1 and 7.  Neither $1472589$ nor $7412589$ is divisible by 7. 

So $def=654$, which eliminates one choice for $ghi$ (so that $g$ is 3 or 7), one possibility for $b$ (so that it must be 8), one possibility for $h$ (so that it must be 2)
\begin{center}
\begin{tabular}{ccccccccccc}
			&	$a$	&	$b$	&	$c$	&	$d$	&	$e$	&	$f$	&	$g$	&	$h$	&	$i$	&	$j$	\\
	1		&		&	x	&		&	x	&	x	&	x	&	x	&	x	&		&	x	\\
	2		&	x	&	x	&	x	&	x	&	x	&	x	&	x	&	2	&	x	&	x	\\
	3		&		&	x	&		&	x	&	x	&	x	&		&	x	&		&	x	\\
	4		&	x	&	x	&	x	&	x	&	x	&	4	&	x	&	x	&	x	&	x	\\
	5		&	x	&	x	&	x	&	x	&	5	&	x	&	x	&	x	&	x	&	x	\\
	6		&	x	&	x	&	x	&	6	&	x	&	x	&	x	&	x	&	x	&	x	\\
	7		&		&	x	&		&	x	&	x	&	x	&		&	x	&		&	x	\\
	8		&	x	&	8	&	x	&	x	&	x	&	x	&	x	&	x	&	x	&	x	\\
	9		&		&	x	&		&	x	&	x	&	x	&	x	&	x	&		&	x	\\
	0		&	x	&	x	&	x	&	x	&	x	&	x	&	x	&	x	&	x	&	0	\\
\end{tabular}
\hspace{1cm}\vline\hspace{1cm}
\begin{tabular}{rll}
	& 		\\
	& 		\\
	& 		\\
	3	&	divides $a+8+c$		& implies 3 divides $a+2+c$\\
	& 		\\
	&		\\
	&		\\
	7	&	divides $a8c654g$	\\
		&	$ghi$ is one of 321, 327, 723, or 729 \\
		&	
		&	
\end{tabular}
\end{center}

We can list the possibilities we need to check,
\begin{center}
\begin{tabular}{ccccccccccc}
			&	$a$	&	$b$	&	$c$	&	$d$	&	$e$	&	$f$	&	$g$	&	$h$	&	$i$	&	$j$	\\
	1		&		&	x	&		&	x	&	x	&	x	&	x	&	x	&		&	x	\\
	2		&	x	&	x	&	x	&	x	&	x	&	x	&	x	&	2	&	x	&	x	\\
	3		&		&	x	&		&	x	&	x	&	x	&		&	x	&		&	x	\\
	4		&	x	&	x	&	x	&	x	&	x	&	4	&	x	&	x	&	x	&	x	\\
	5		&	x	&	x	&	x	&	x	&	5	&	x	&	x	&	x	&	x	&	x	\\
	6		&	x	&	x	&	x	&	6	&	x	&	x	&	x	&	x	&	x	&	x	\\
	7		&		&	x	&		&	x	&	x	&	x	&		&	x	&		&	x	\\
	8		&	x	&	8	&	x	&	x	&	x	&	x	&	x	&	x	&	x	&	x	\\
	9		&		&	x	&		&	x	&	x	&	x	&	x	&	x	&		&	x	\\
	0		&	x	&	x	&	x	&	x	&	x	&	x	&	x	&	x	&	x	&	0	\\
\end{tabular}
\hspace{1cm}\vline\hspace{1cm}
\begin{tabular}{rll}
	& 		\\
	& 		\\
	& 		\\
	3	&	divides $a+2+c$		& implies $ac$ is one of 13, 19, 31, 79, 91, or 97 \\
	& 		\\
	&		\\
	&		\\
	7	&	divides $a8c654g$	\\
		&	$ghi$ is one of 321, 327, 723, or 729 \\
		&	
		&	
\end{tabular}
\end{center}
and $g$ is either 3 or 7, while $i$ is the final unused odd number.

Now we begin exhaustive checking.

\begin{itemize}
	\item If $ac$ is 13 or 31 then $ghi$ must be 729.  1836547 is not divisible by 7 but 3816547 is!

	\item If $ac$ is 79 or 97 then $ghi$ must be 321.  Neither 7896543 nor 9876543 is divisible by 7.

	\item If $ac$ is 19 or 91 then $ghi$ must be 327 or 723.  None of the 4 possible numbers are divisible by 7.
\end{itemize}
Therefore, the only choice is $ac=31$, $ghi=729$,

The solution is
\begin{center}
\begin{tabular}{ccccccccccc}
			&	$a$	&	$b$	&	$c$	&	$d$	&	$e$	&	$f$	&	$g$	&	$h$	&	$i$	&	$j$	\\
	1		&	x	&	x	&	1	&	x	&	x	&	x	&	x	&	x	&	x	&	x	\\
	2		&	x	&	x	&	x	&	x	&	x	&	x	&	x	&	2	&	x	&	x	\\
	3		&	3	&	x	&	x	&	x	&	x	&	x	&	x	&	x	&	x	&	x	\\
	4		&	x	&	x	&	x	&	x	&	x	&	4	&	x	&	x	&	x	&	x	\\
	5		&	x	&	x	&	x	&	x	&	5	&	x	&	x	&	x	&	x	&	x	\\
	6		&	x	&	x	&	x	&	6	&	x	&	x	&	x	&	x	&	x	&	x	\\
	7		&	x	&	x	&	x	&	x	&	x	&	x	&	7	&	x	&	x	&	x	\\
	8		&	x	&	8	&	x	&	x	&	x	&	x	&	x	&	x	&	x	&	x	\\
	9		&	x	&	x	&	x	&	x	&	x	&	x	&	x	&	x	&	9	&	x	\\
	0		&	x	&	x	&	x	&	x	&	x	&	x	&	x	&	x	&	x	&	0	\\
\end{tabular}
\hspace{1cm}\vline\hspace{1cm}
\begin{tabular}{rll}
	& 		\\
	1	&	divides $3$,					\\
	2	&	divides $38$,					\\
	3	&	divides $381$,					\\
	4	&	divides $3816$,					\\
	5	&	divides $38165$,				\\
	6	&	divides $381654$,				\\
	7	&	divides $3816547$,				\\
	8	&	divides $38165472$,				\\
	9	&	divides $381654729$, and		\\
	10	&	divides $3816547290$.
\end{tabular}
\end{center}
